% !TeX spellcheck = en_GB
% !TeX root = ../main-scirep.tex
\section{Discussion}
\label{section:discussion}

This paper investigates the growth of federal legislation in two industrial countries over a period of 25 years. 
As such, it is limited in \emph{geographic} scope (United States and Germany), 
\emph{temporal} scope (1994--2018), and \emph{institutional} scope (legislative bodies on the federal level). 
This makes it hard to assess to which extent the growth we observe is \emph{particular} to our data or rather \emph{universal}. 
The trend we identify applies to the recent history of federal legislation in at least two countries, the United States and Germany, 
and our findings in one country provide context for our findings in the other. 
Thus, we can establish that the growth we find is not a singular phenomenon, but we can only guess how it relates to the trends we might find in the legal document networks of other countries, time frames, or institutions.

The document networks that are most closely related to legislative networks are networks of regulations (produced by executive agencies) and networks of judicial decisions (produced by courts)---%
and for all of them, growth statistics that are directly comparable to ours are lacking. 
Some growth statistics are known for patent citation networks \cite{strandburg:2006, strandburg:2009,torrance2017,uspto2020},
where, e.g., the number of patents granted annually by the United States Patent and Trademark Office has roughly tripled in the 25 years from 1994 to 2018 \cite{uspto2020}. 
Since the generating processes of patent citation networks are very different from those of legislative networks 
(patent applicants need to cite prior art in their filings, patent examiners can add further citations, and too much prior art might risk patentability) 
and the units of analysis are not the same (structural elements in legislative networks vs. individual patents in patent citation networks), 
however, this result has little bearing on our findings. 
For similar reasons, comparing our findings with results on non-legal document networks, 
such as the World Wide Web or scholarly networks, is potentially misleading.
To put our findings into perspective, 
extending the scope of our data to other legal document networks is therefore an important direction for future work.
For example, investigations in the following directions are supported by our legal network data model:
\begin{enumerate}
	\item Analysing legislative activity on levels above and below the federal level
	and comparing the results with our findings will advance the search for invariants that characterize the development of legislative systems. 
	It can also help us understand the division of labour within the legislative pyramid (e.g., the federal, the state, and the local level).
	Does state law grow even faster than federal law? 
	If so, are the growth mechanisms similar or different? 
	How do the answers to these questions depend on the allocation of legislative competencies?
	
	\item Integrating documents from the executive and judicial branches of government with our datasets could help us explore how different parts of the legal system interact. 
	How does the evolution of a legislative network compare to that of a network of administrative regulations, a network of executive orders, or a network of judicial decisions? 
	In what areas of law is the development driven by the executive or the judiciary, rather than the legislative?
	What does this tell us about the distribution of power between the different branches of government?
	
	\item Combining our legislative network data 
	with data collected in other fields of quantitative social science 
	might improve our understanding of the interaction between legal rules and other rule sets that impact the behaviour of individuals and societies.
	When, where, and how do legislative changes impact how people behave on the ground? 
	When, where, and how do changes in how people behave prompt legislative changes? 
	In other words: What \emph{causal} relationships can we establish between legal change and societal change? 
	These questions are inherently multidisciplinary, 
	and to separate causes from confounders, 
	legal network data would need to be combined with data reflecting public sentiment (e.g., social media data or public news data) and data reflecting individual or collective choices (e.g., 
	financial network data, company reports, or economic panel data on households, firms, and non-governmental organisations). 
	Similarly, a multi-pronged strategy could be pursued to investigate the relationships between legal change and \emph{technological} change. 
	Here, combining legal network data with patent citation network, patent litigation, and R\&D investment data appears to be particularly promising.
\end{enumerate}

Methodologically, 
our approach emphasizes the structural features of legislative texts.
In particular, for the results we report in this work, 
the content of the legal texts has been only of indirect interest, 
e.g., as reflected in raw token counts or in reference structures that characterize legal topics.
As demonstrated in Section~\ref{subsection:reorganisation}, however,
qualitative analyses of the legal rules contained in our document networks can yield further insights, 
and this opens opportunities for normative legal research in areas such as comparative law and legal theory \cite{armour2009,spamann2009,cabrelli2015}.
In these legal disciplines, 
the United States and Germany are usually classified as following different legal traditions, also referred to as \emph{legal families},
and the categorization, though commonly accepted, has not been corroborated by empirical studies \cite{zweigertkoetz1998,husa2016,siems2016,glenn2019}.

Last but not least, the findings reported in this paper are based on a set of choices for methods and parameters.
For example, we examine growth by analysing year-to-year net gain of tokens, 
as this difference can be determined reliably.
The amount of legislative activity, however, is likely much higher (e.g., deletions and additions cancel out from the net gain perspective), 
and developing tools that allow for a fine-grained accounting of legislative changes constitutes an interesting research direction.
While we explored our model space extensively (as detailed in Section~4 of the~\suppi),
the parametrisation of the clustering required numerous decisions based on our experience and familiarity with the subject matter.
Other parametrisations are possible, 
and they might be needed in other analytical settings.
In particular, future work could examine selected parts of our data in greater detail, 
zoom in on a particular legal topic, 
and therefore choose very different parameters to operate at a higher level of resolution.
